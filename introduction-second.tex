\section{Beispielanwendung}

Um die aufgeführten asynchronen Operationen in der praxisnahen Anwendung zu sehen, wird für diese Bachelorarbeit eine Beispiel-Applikation aufgesetzt. Diese Applikation wird aus zwei Schichten bestehen:

\subsection{Angular (Frontend)}

\subsection{Google Firebase (DaaS)}
Um praxisnahe asynchrone Verarbeitungsprozesse zu simulieren wie z.B. Daten von einem Server abrufen oder Daten an einem Server hervorbringen, braucht das Beispielprojekt eine Schnittstelle zu einer Datenbank. Um dies so einfach wie möglich zu gestalten (ohne eine komplette Backend-Applikation mit Datenbankanbindung zu bauen), bietet Google mit Firebase eine Plattform die clientseitig eine Kommunikation zwischen Frontend und DaaS ermöglicht. Firebase bietet standartmäßig eine echtzeit \textit{(englisch \glqq{}realtime\grqq{})} Datenbank, die treffend für die Verarbeitung von Observables- und Promises-Operationen ist.

\begin{figure}[H]
\centering
\includegraphics[width=3.5cm, height=1cm]{firebase-logo}
\caption{Firebase-Logo\cite{firebase}}
\end{figure}

Um das Beispielprojekt mit Firebase zu initialisieren, wird ein Google-Account benötigt.