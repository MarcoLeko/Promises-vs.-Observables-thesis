\setcounter{secnumdepth}{1}

\section{Einführung}

Javascript ist im Aufwind mehr denn je. Sie ist die meist genutzte Programmiersprache unter Entwicklern. Laut der diesjährigen Umfrage von Stackoverflow belegte sie zum sechsten mal in Folge den ersten Platz\cite{programming-language-survey}. Doch woher kommt dieser Erfolg? \\

\noindent
Wenn es um das Web geht, war Javascript als Skriptsprache schon immer Vorreiter. Doch diese Sprache hat seine Stärken auch in der Vielfalt seiner Einsatzmöglichkeiten. Wie mit dem Framework Node.js, dass ab 2009 in Erscheinung getreten ist, war es möglich Serverseitigen Javascript-Code zu implementieren. Dies hatte zur Folge, dass man sich nicht mit multiplen Programmiersprachen auseinandersetzen musste. Man hatte eine Vereinheitlichung in seiner Code-Basis gefunden. Zudem bietet Node.js mit npm als Paketmanager die größte Auswahl an Open-Source-Libraries. Ein weiterer Punkt ist die Multi-Plattform-Kompatibilität, die Javascript z.B. mit Angular offenbart. Mit diesem Framework kann man mit einem Projekt sowohl Web-Anwendungen als auch Apps für mobile Geräte entwickeln. Diese Hybrid-Applikationen werden Progressive Web Apps genannt und können durch optimiertes Caching auch im Offline Modus betrieben werden. Das ist nur möglich, da Javascript's Ausführung vom Betriebssystem unabhängig ist. Und genau das bietet auch Typescript. Da Typescript ein \textit{Superset} von Javascript ist, kann jeder Javascript Code auch in Typescript interpretiert werden. Der Unterschied von Typescript zu Javascript ist die statische Typisierung. Das bedeutet, dass Variablen, Methoden und Funktionsparameter im Vorfeld mit einem Typ versehen werden. Aufgrund dessen können Bugs zur Laufzeit schneller erkannt werden. Zudem lässt sich der Code durch die Kapselung in Objekten leicht modular aufbauen und dokumentieren.\\

\noindent
Doch mit der Vielfalt steigt auch die Komplexität der Sprache. Schon zum vierten Jahr in Folge erschien eine neue Version von Javascript, die neueste Features mit sich bringt. Beispielsweise ermöglichen Promises und async await einen weiteren Ansatz zur asynchronen Verarbeitung. Somit werden Applikation, die ereignisgesteuert operieren, neue Möglichkeiten geboten. Mit Callbacks, Promises und Streams bekommen Entwickler verschiedene Optionen der asynchronen Datenverarbeitung zur Verfügung. Die Einsatzmöglichkeiten in einem Projekt können variieren, jedoch welcher Ansatz in welcher Situation sinnvoller erscheint, ist besonders für Entwickler mit wenig Erfahrung schwer zu erkennen.

\subsection{Zielsetzung}

Ziel der Arbeit ist es die Betrachtung und Aufbereitung des Einsatzes von Sprachmitteln zur asynchronen Verarbeitung in Typescript in einer Form, die Einsteigern in die Thematik hilft, die unterschiedlichen Konzepte voneinander abzugrenzen und richtig einzusetzen. Dazu soll asynchrone Verarbeitung insgesamt anhand brauchbarer und für den Einsatz von Typescript typischer Szenarien motiviert werden. Es werden die zur Verfügung stehenden Sprachmittel wie Callbacks, Promises und Observables mit ihren jeweiligen Features vorgestellt. Nach dem Lesen dieser Arbeit soll der Leser ein gewisses Verständnis dafür gewonnen haben, welche der vorgestellten Sprachmitteln für welchen Anwendungsfall besser geeignet ist.\\

\noindent
Zielgruppe dieser Arbeit sind Personen, die Grundkenntnisse in der Sprache Javascript/Typescript vorweisen können. Des Weiteren sollte man von dem Modell der asynchronen Datenverarbeitung wie Callbacks, Promises und Oberservables schon mal gehört haben, da diese in der Arbeit gegenübergestellt werden.

\subsection{Aufbau}

In dieser Arbeit werden die jeweiligen Ausarbeitungen der Ansätze mit Code-Beispielen unterstützt. Neben simulierten Anwendungsbeispielen zum Einstieg werden auch praxisnahe Beispiele wie z.B. Aufrufe an einer REST-API dargestellt. Das Code-Projekt lässt sich fachlich in den Modulen Callbacks, Promises, Async Await und RxJS unterteilen. Die Module haben jeweils eine introduction.ts Datei zur Einführung, eine stories.ts Datei, die entweder Funktionen oder Klassen anbietet und eine stories-usage.ts Datei, die diese Klassen oder Funktionen ausführt. In den jeweiligen Sektionen wird separat gezeigt, wie die einzelnen Dateien ausgeführt werden. Das Repository für das Projekt ist zu finden unter: 

\begin{center}
\url{https://github.com/MarcoLeko/async-patterns.git}
\end{center}

\noindent
Zu beachten ist, dass vor dem Ausführen des Projektes \textbf{Node.js} auf dem Rechner installiert sein muss, um den integrierten Paket-Manager \textbf{npm} nutzen zu können. Da \textbf{RxJS} nicht von vornherein von Javascript mitgeliefert wird, wird eine sog. \glqq Third-Party Library\grqq{} dafür benötigt. Sobald npm installiert ist, muss in dem Stammverzeichnis des Projekts \textbf{npm install} ausgeführt werden, um die jeweiligen Libraries herunterzuladen. Node.js kann man unter folgendem Link herunterladen:

\begin{center}
\url{https://nodejs.org/en/}
\end{center}

\section{Grundlagen}

\subsection{Typescript}
Typescript. Bereits der Name sagt, was diese Sprache ausmacht. Sie ist eine typisierte Form der Skriptsprache Javascript. Besonders Javascript-Entwickler sträubten sich Anfangs mit der Sprache auseinanderzusetzen. Die Stärke von Javascript liegt in ihrer Flexibilität und der Tatsache nicht mehr Code schreiben zu müssen, als man braucht. Doch den Kompromiss, den man mit Typescript eingeht, zahlt sich am Ende mit dem Zuwachs der Projektgröße aus. Da der Typescript Kompiler bereits Fehler beim Kompilieren erkennt, stellt sich die Frage, ob man lieber Fehler beim Entwickeln oder in der Produktion eines Projekts entdecken möchte. Das \textbf{Tooling} wird mit Typescript ebenfalls erheblich vereinfacht. Wenn man mit Typ-Annotationen oder mit Libraries, die streng typisiert sind, arbeitet, wird der Code von der Entwicklungsumgebung automatisch dokumentiert. Das heißt beim Entwickeln mit einem Code-Editor wird nicht zusätzlich verlangt, die Dokumentation der benötigten Library zu öffnen, um jeden einzelnen Rückgabetyp der Methode zu sehen. Wenn man nun von einem reinen Javascript-Hintergrund kommt, gibt es so gut wie keine Lernkurve die Sprache zu bewältigen. Das liegt daran, dass jeder valide Javascript Code auch in Typescript ausführbar ist. Der Lernprozess steigt mit zunehmender Nutzung.

\subsubsection{Beispiel}

\begin{figure}[H]
\begin{lstlisting}[basicstyle=\small]
class Greeter {
    greeting: string;
    constructor (message: string) {
        this.greeting = message;
    }
    greet() {
        return "Hello, " + this.greeting;
    }
}  
\end{lstlisting}
\caption{Angabe einer Typescript Beispiel Klasse \cite{typescript-example}}
\end{figure}

\begin{figure}[H]
\begin{lstlisting}[basicstyle=\small]
var Greeter = (function () {
    function Greeter(message) {
        this.greeting = message;
    }
    Greeter.prototype.greet = function () {
        return "Hello, " + this.greeting;
    };
    return Greeter;
})(); 
\end{lstlisting}
\caption{Überführung in Javascript \cite{typescript-example}}
\end{figure}

\noindent
Typescript selbst kann nicht selbständig ausgeführt werden. Das heißt in einem Browser oder auf einem Server wird unter Umständen immer noch Javascript genutzt. Mit dem Typescript Compiler werden daher Dateien mit dem Suffix *.ts in *.js überführt. Im oberen Code-Schnipsel wurden die Variablen und die Klassenmethoden nach Typescript-Standard deklariert. Diese Typen werden beim transpilieren in Javascript ignoriert. Der Kompiler prüft dann, ob bei der Übergabe des Parameters in dem Konstruktor ein String Wert eingesetzt wird. Wenn nicht, wird dies als ein Fehler erkannt. In der vom Kompiler auf Javascript übersetzten Version werden die erstellten Klassen und Typen vollständig eliminiert. Was verbleibt ist die Übersetzung der Methode greet() und des Konstruktors. Zudem werden auch nicht direkt deklarierte (implizite) Typen übersetzt. Wie in diesem Fall wird erkannt, dass die Methode greet() ein Wert vom Typ String zurückgibt. Ungleich wie mit Java oder C\# gewährleistet Typescript aus der streng typisierten Welt rauszufahren.

\begin{figure}[H]
\begin{lstlisting}[basicstyle=\small]
let count: any = 23;

count = 'Oops transformed to string!'
\end{lstlisting}
\caption{Wenn unbekannte Objekte von einem API-Aufruf verarbeitet werden, können diese als any deklariert werden.}
\end{figure}

\noindent
Der Kompiler würde hierbei keinen Fehler anzeigen. Obwohl im Idealfall auf die \textbf{any} Notation verzichtet werden sollte, wird dennoch Javascripts Flexibilität geboten. 

\subsubsection{Kompiler}
Ein großer Vorteil von Typescript ist die Möglichkeit auf verschiedene Javascript/Ecmascript Versionen zu transpilieren. Wenn bestimmte Browser Javascripts neueste Feature noch nicht unterstützen, kann das Kompilat auf ältere Ecmascript Versionen übersetzt werden. Diese Konfiguration wird mit Hilfe einer \textbf{tsconfig.json} Datei bewältigt. Die Auflistung einer solchen Datei zeigt, dass es sich um das Stammverzeichnis des Projekts handelt. Hier können die Kompiler-Optionen für Typescript gesetzt oder Dateien von der Übersetzung ein- oder ausgeschlossen werden. Ein Beispiel für die Konfiguration einer solchen Datei könnte wie folgt aussehen: 

\begin{figure}[H]
\begin{lstlisting}[basicstyle=\small]
{
  "compilerOptions": {
    "target": "esnext",
    "module": "commonjs",
    "lib": ["dom", "es2018"],
    "typeRoots": [
      "node_modules/@types"
    ]
  }
}
\end{lstlisting}
\caption{Die Auflistung einer tsconfig.json Datei}
\end{figure}

\noindent
Mit dem Wert “target” wird die ausgeführte Javascript Version eingestellt. Wenn der Wert nicht bestimmt ist, wird standardmäßig nach es3 übersetzt. Um stets die aktuellste Version zu nutzen, wird für das Projekt \textbf{esnext} angegeben. Typescript wird im Terminal unter dem Verzeichnis einer Typescript-Datei mit \textit{tsc Dateiname.ts} kompiliert. In den für diese Arbeit genutzten Code-Beispielen werden die Startskripte von npm ausgeführt.

\begin{figure}[H]
\begin{lstlisting}[basicstyle=\small]
  ...
  "scripts": {
    "start": "webpack --watch",
    "build": "webpack"
  },
  ...
\end{lstlisting}
\caption{Die Befehle sind unter der \textbf{package.json} Datei zu finden.}
\label{npm-ex-script}
\end{figure}

\noindent
Wie man in Abbildung \ref{npm-ex-script} sehen kann, führt \textbf{npm run start} nicht tsc aus, sondern \textbf{webpack --watch} aus. Webpack ist ein sog. \textbf{Module-Bundler}. Das Projekt ist nicht nur fachlich, sondern auch technisch in Modulen gegliedert. Diese Module sollten nach Möglichkeit einzeln ausgeführt werden. Ein Module-Bundler läuft während des Build-Prozesses und erfüllt im Wesentlichen zwei Aufgaben. Den Code mit dem Typescript-Kompiler in Javascript zu transpilieren und die daraus resultierenden Kompilate in einer zentralen Datei (main.js) zu bündeln. Um das zu erreichen, nimmt Webpack eine Eingangsdatei als Einstiegspunkt an und bündelt den Inhalt inklusive aller involvierten Importe. Welchen Mehrwert hat Webpack also für das Projekt? \\

\noindent
Beim Ausführen der Beispiele muss nicht jedes mal die jeweilige TS-Datei in eine JS-Datei überführt und das erstellte Kompilat daraufhin in die HTML-Datei referenziert werden. Die HTML-Datei hat nur das Bündel main.js als Referenz und je nach definiertem Einstiegspunkt in der \textbf{webpack.config.js} wird ein neues Bündel erzeugt. Mit der Einstellung \textbf{--watch} werden Veränderung des Codes in Echtzeit kompiliert. Sollten nun Änderungen im Code vorgenommen werden, erstellt Webpack ein neues Bündel. Nach dem Kompilieren muss nur noch die HTML-Seite geöffnet/aktualisiert werden. Um mehr von Typescript und dessen Nutzung mit Webpack zu erfahren, findet unter folgendem Link mehr dazu:

\begin{center}
\url{https://webpack.js.org/guides/typescript/} 
\end{center}

\subsection{Terminologie}

Um die Abgrenzung von verschiedenen asynchronen Sprachmitteln zu beherrschen, muss vorerst geklärt werden, was Asynchronität bedeutet.\\

\noindent
Der zentrale Teil eines Computers, welches Programme und einzelne Schritte zum Ausführen bringt, ist der Prozessor. Die Geschwindigkeit in der eine Schleife von einem Programm ausgeführt wird, hängt von der Geschwindigkeit des Prozessors ab. Programme interagieren jedoch mit Operationen außerhalb des Zuständigkeitsbereiches eines Prozessors, wie z.B. die Kommunikation über ein Netzwerk oder das Abfragen von Daten von der Festplatte. Solche Operationen hängen von anderen Ressourcen ab und brauchen Zeit. In solchen Szenarien wäre es unvorteilhaft, wenn der Prozessor, anstatt andere Operationen in der Zwischenzeit auszuführen, im Leerlauf stecken würde. Genau für solche Aufgaben ist das Betriebssystem da. Das Betriebssystem sorgt dafür, dass Kapazitäten des Prozessors auf parallel laufende Programme wechselt, während es auf die Antwort von ausgeführten Operationen wartet. Es entstehen dabei verschiedene \glqq Threads\grqq{}. Und hier kommt die Asynchronität ins Spiel:

\subsubsection{Asynchronität}

Ein \textbf{asynchrones} Programmiermodell erlaubt multiple Abläufe zum selben Zeitpunkt. Wenn eine Aktion ausgeführt wird, läuft das Programm in einem anderen \glqq Thread\grqq{} weiter. Sollte die Operation fertig gestellt sein, wird das Programm informiert und liefert das Ergebnis zurück. Ein Beispiel dafür wäre das Suchen von Dateien auf einer Festplatte\cite{asynchronitaet}.

\subsubsection{Synchronität}

In einem \textbf{synchronen} Programmiermodell entstehen Abläufe sequenziell. Wenn eine Funktion mit einer zeitintensiven Aktion abgerufen wird, wird das Ergebnis erst nach dem Beenden der Operation zurückgegeben. In diesem Zeitintervall werden keine Nebenoperationen ausgeführt\cite{asynchronitaet}.\\

\noindent
Man kann ein synchrones und asynchrones Modell mit einem kleinem Beispiel vergleichen: Eine Anwendung die zwei Ressourcen aus einem Netzwerk anfragt und die Ergebnisse miteinander kombiniert. In einer synchronen Umgebung würde man dies mit einem Funktionsaufruf nach dem anderen bewältigen. Das hat den Nachteil, dass die zweite Funktion erst dann ausführt, wenn die erste fertig ist. Die Gesamtzeit der Ausführung ist die Summe der Antwortzeiten der beiden Anfragen. Die Lösung dafür ist ein synchrones Programmverhalten mit mehreren kontrollierten Threads. Ein Thread ist ein Ausführungsstrang, der seine eigene Zeitleiste besitzt und innerhalb dieser Operationen ausführt. Ein zweiter Thread könnte zum selben Zeitpunkt die Funktion aufrufen und wenn beide Threads das Ergebnis geliefert bekommen, resynchronisieren sie sich und kombinieren ihre Ergebnisse\cite{asynchronitaet}. Im folgenden Diagramm wird ein synchrones Modell mit einem Thread, mit zwei Threads und ein asynchrones Modell gegenübergestellt.

\begin{center}
\begin{figure}[H]
\includegraphics[width=12cm]{synchron-vs-asynchron-diagram}
\caption{Die blauen Linien repräsentieren die Zeit, in der das Programm ausführt und die roten Linien repräsentieren die Zeit, in der auf die Antwort gewartet wird.}
\end{figure}
\end{center}

\noindent
Im synchronen Modell ist die Zeit der Netzwerkanfrage ein Teil der Zeitleiste. Im asynchronen Modell hingegen führt das Starten einer asynchronen Operation zu einer neuen Zeitleiste. Zusammenfassend kann man also sagen, dass in dem synchronen Modell explizit auf die Aktionen gewartet wird, und in dem Asynchronen implizit.